%
%

\documentclass[11pt,a4paper,oneside]{book}

\usepackage[spanish]{babel}
%-------------------------------------------------------------------
% Esto es para poder escribir acentos directamente:
\usepackage[utf8]{inputenc}
\usepackage[T1]{fontenc}
%-----------------------------------------------------------------------
% para uso en modo matamático
\usepackage{amssymb}
\usepackage{amsthm}
\usepackage{amsmath}

\usepackage{verbatim} % para el codigo
\usepackage{listings} % para codigo con formateado mas complejo
\usepackage{graphicx} 
\graphicspath{{Figuras/}} % se fija el camino para las figuras

\usepackage{longtable} % Tablas largas

\begin{document}

\tableofcontents % indice de contenidos
 
\chapter{Introducción}

\chapter{Code Smells}

\section{Monstruos - Bloaters}

Bloaters are code, methods and classes that have increased to such gargantuan proportions that they are hard to work with. Usually these smells do not crop up right away, rather they accumulate over time as the program evolves (and especially when nobody makes an effort to eradicate them).

\subsection{Método largo -   Long Method}
    Large Class
    Primitive Obsession
    Long Parameter List
    Data Clumps




\section{Object-Orientation Abusers}

All these smells are incomplete or incorrect application of object-oriented programming principles.

    Switch Statements
    Temporary Field

    Refused Bequest
    Alternative Classes with Different Interfaces





\section{Change Preventers}

These smells mean that if you need to change something in one place in your code, you have to make many changes in other places too. Program development becomes much more complicated and expensive as a result.

    Divergent Change
    Shotgun Surgery

    Parallel Inheritance Hierarchies




\section{Prescindibles - Dispensables}

A dispensable is something pointless and unneeded whose absence would make the code cleaner, more efficient and easier to understand.

    Comments
    Duplicate Code
    Lazy Class

    Data Class
    Dead Code
    Speculative Generality



\section{Couplers}

All the smells in this group contribute to excessive coupling between classes or show what happens if coupling is replaced by excessive delegation.

    Feature Envy
    Inappropriate Intimacy
    Message Chains
    Middle Man

    Incomplete Library Class


\section{Other Smells}

Below are the smells which do not fall into any broad category.

Incomplete Library Class

Sooner or later, libraries stop meeting user needs. The only solution to the problem – changing the library – is often impossible since the library is read-only.



\chapter{Técnicas de refactorización}


\section{Composing methods}

Much of refactoring is devoted to correctly composing methods. In most cases, excessively long methods are the root of all evil. The vagaries of code inside these methods conceal the execution logic and make the method extremely hard to understand – and even harder to change.

The refactoring techniques in this group streamline methods, remove code duplication, and pave the way for future improvements.

    Extract Method
    Inline Method
    Extract Variable
    Inline Temp
    Replace Temp with Query
    Split Temporary Variable
    Remove Assignments to Parameters
    Replace Method with Method Object
    Substitute Algorithm


\section{Moving Features between Objects}

Even if you have distributed functionality among different classes in a less-than-perfect way, there is still hope.

These refactoring techniques show how to safely move functionality between classes, create new classes, and hide implementation details from public access.

    Move Method
    Move Field
    Extract Class
    Inline Class
    Hide Delegate
    Remove Middle Man
    Introduce Foreign Method
    Introduce Local Extension



\section{Organizing Data}

These refactoring techniques help with data handling, replacing primitives with rich class functionality. Another important result is untangling of class associations, which makes classes more portable and reusable.

    Self Encapsulate Field
    Replace Data Value with Object
    Change Value to Reference
    Change Reference to Value
    Replace Array with Object
    Duplicate Observed Data
    Change Unidirectional Association to Bidirectional
    Change Bidirectional Association to Unidirectional
    Replace Magic Number with Symbolic Constant
    Encapsulate Field
    Encapsulate Collection
    Replace Type Code with Class
    Replace Type Code with Subclasses
    Replace Type Code with State/Strategy
    Replace Subclass with Fields




\section{Simplifying Conditional Expressions}

Conditionals tend to get more and more complicated in their logic over time, and there are yet more techniques to combat this as well.

    Decompose Conditional
    Consolidate Conditional Expression
    Consolidate Duplicate Conditional Fragments
    Remove Control Flag

    Replace Nested Conditional with Guard Clauses
    Replace Conditional with Polymorphism
    Introduce Null Object
    Introduce Assertion




\section{Simplifying Method Calls}

These techniques make method calls simpler and easier to understand. This, in turn, simplifies the interfaces for interaction between classes.

    Rename Method
    Add Parameter
    Remove Parameter
    Separate Query from Modifier
    Parameterize Method
    Replace Parameter with Explicit Methods
    Preserve Whole Object
    Replace Parameter with Method Call
    Introduce Parameter Object
    Remove Setting Method
    Hide Method
    Replace Constructor with Factory Method
    Replace Error Code with Exception
    Replace Exception with Test




\section{Dealing with Generalisation}

Abstraction has its own group of refactoring techniques, primarily associated with moving functionality along the class inheritance hierarchy, creating new classes and interfaces, and replacing inheritance with delegation and vice versa.

    Pull Up Field
    Pull Up Method
    Pull Up Constructor Body
    Push Down Method
    Push Down Field
    Extract Subclass
    Extract Superclass
    Extract Interface
    Collapse Hierarchy
    Form Template Method
    Replace Inheritance with Delegation
    Replace Delegation with Inheritance

\section{Otras refactorizaciones de Fowler}
% https://refactoring.com/catalog/
Change Function Declaration
Change Signature 
Inline Function
Inline Variable
Introduce Explaining Variable
Introduce Special Case
Move Function
Move Statements into Function
Move Statements to Callers
Parameterize Function
Remove Dead Code
Remove Flag Argument
Remove Subclass
Rename Field
Rename Function 
Rename Variable
Replace Command with Function
Replace Constructor with Factory Method
Replace Control Flag with Break
Replace Derived Variable with Query
Replace Exception with Precheck
Replace Function with Command
Replace Inline Code with Function Call
Replace Loop with Pipeline
Replace Magic Literal
Replace Parameter with Query
Replace Primitive with Object
Replace Query with Parameter
Replace Record with Data Class
Replace Subclass with Delegate
Replace Superclass with Delegate
Replace Type Code with Class
Return Modified Value
Slide Statements
Split Loop
Split Phase
Split Variable



\chapter*{Correspondencia entre refactorizaciones}

\begin{longtable}{|p{200pt}|p{200pt}|}
\footnotesize
Categoría refactorización & Categoría Fowler\\ 
\hline
    Add Parameter & Add Parameter\\ 
    Change Bidirectional Association to Unidirectional & \\ 
    Change Reference to Value & Change Reference to Value\\ 
    Change Unidirectional Association to Bidirectional & \\ 
    Change Value to Reference & Change Value to Reference\\ 
    Collapse Hierarchy & Collapse Hierarchy\\ 
 & Combine Functions into Class\\ 
 & Combine Functions into Transform\\ 
    Consolidate Conditional Expression & Consolidate Conditional Expression\\ 
    Consolidate Duplicate Conditional Fragments & Consolidate Duplicate Conditional Fragments\\ 
    Decompose Conditional & Decompose Conditional\\ 
    Duplicate Observed Data & \\ 
    Encapsulate Collection & Encapsulate Collection\\ 
    Encapsulate Field & Encapsulate Field \\ 
 & Encapsulate Record\\ 
 & Encapsulate Variable\\ 
    Extract Class & Extract Class\\ 
    Extract Interface & Extract Function\\ 
    Extract Method & Extract Method\\ 
    Extract Subclass & Extract Subclass \\ 
    Extract Superclass & Extract Superclass\\ 
    Extract Variable & Extract Variable\\ 
    Form Template Method & \\ 
    Hide Delegate & Hide Delegate\\ 
    Hide Method & \\ 
    Inline Class & Inline Class\\ 
    Inline Method & Inline Method\\ 
    Inline Temp & Inline Temp\\ 
    Introduce Assertion & Introduce Assertion\\ 
    Introduce Foreign Method & \\ 
    Introduce Local Extension & \\ 
    Introduce Null Object & Introduce Null Object\\ 
    Introduce Parameter Object & Introduce Parameter Object\\ 
    Move Field & Move Field\\ 
    Move Method & Move Method\\ 
    Parameterize Method & Parameterize Method\\ 
    Preserve Whole Object & Preserve Whole Object\\ 
    Pull Up Constructor Body & Pull Up Constructor Body\\ 
    Pull Up Field & Pull Up Field\\ 
    Pull Up Method & Pull Up Method\\ 
    Push Down Field & Push Down Field\\ 
    Push Down Method & Push Down Method\\ 
    Remove Assignments to Parameters & Remove Assignments to Parameters \\ 
    Remove Control Flag & Remove Control Flag\\ 
    Remove Middle Man & Remove Middle Man\\ 
    Remove Parameter & Remove Parameter \\ 
    Remove Setting Method & Remove Setting Method\\ 
    Rename Method & Rename Method\\ 
    Replace Array with Object & \\ 
    Replace Conditional with Polymorphism & Replace Conditional with Polymorphism\\ 
    Replace Constructor with Factory Method & Replace Constructor with Factory Function\\ 
    Replace Data Value with Object & Replace Data Value with Object\\ 
    Replace Delegation with Inheritance & \\ 
    Replace Error Code with Exception & Replace Error Code with Exception\\ 
    Replace Exception with Test & Replace Exception with Test\\ 
    Replace Inheritance with Delegation & Replace Inheritance with Delegation\\ 
    Replace Magic Number with Symbolic Constant & Replace Magic Number with Symbolic Constant\\ 
    Replace Method with Method Object & Replace Method with Method Object\\ 
    Replace Nested Conditional with Guard Clauses & Replace Nested Conditional with Guard Clauses\\ 
    Replace Parameter with Explicit Methods & Replace Parameter with Explicit Methods\\ 
    Replace Parameter with Method Call & Replace Parameter with Method\\ 
    Replace Subclass with Fields & Replace Subclass with Fields\\ 
    Replace Temp with Query & Replace Temp with Query\\ 
    Replace Type Code with Class & \\ 
    Replace Type Code with State/Strategy & Replace Type Code with State/Strategy\\ 
    Replace Type Code with Subclasses & Replace Type Code with Subclasses\\ 
    Self Encapsulate Field & Self-Encapsulate Field\\ 
    Separate Query from Modifier & Separate Query from Modifier\\ 
    Split Temporary Variable & Split Temp\\ 
    Substitute Algorithm &     Substitute Algorithm\\
\end{longtable}





\chapter{Caso 1}


\chapter{Caso 2}


\chapter {Caso 3}







\end{document}


